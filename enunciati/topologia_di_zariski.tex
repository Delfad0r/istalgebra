\section{Topologia di Zariski}
\begin{definition}
Sia $A$ un anello. La topologia di Zariski è la topologia su $\Spec A$ i cui chiusi sono
$$
\VV(I)=\{\pp\in\Spec A:\pp\sups I\}
$$
al variare di $I\ideal A$.
\end{definition}
\begin{proposition}
Sia $A$ un anello. Una base della topologia di Zariski è
$$
\{(\Spec A)_a:a\in A\}.
$$
\end{proposition}
\begin{definition}
Sia $\map{f}{A}{B}$ un omomorfismo di anelli. Si definisce l'applicazione $\map{f^*}{\Spec B}{\Spec A}$ tale che $f^*(\qq)=f^{-1}(\qq)$.
\end{definition}
\begin{proposition}
\leavevmode
\begin{itemize}
\item $\map{f^*}{\Spec B}{\Spec A}$ è continua.
\item $(f\circ g)^*=g^*\circ f^*$.
\item $(\1_A)^*=1_{\Spec A}$
\end{itemize}
\end{proposition}
\begin{corollary}
Esiste un funtore controvariante
\functor{\cat{Ring}}{\cat{Top}}{A}{\Spec A}{f}{f^*}
\end{corollary}
\begin{definition}
Sia $\map{f}{A}{B}$ un omomorfismo di anelli. Si dice che $f$ soddisfa il \emph{lying over} se $\map{f^*}{\Spec B}{\Spec A}$ è suriettiva.
\end{definition}
\begin{definition}
Sia $\map{f}{A}{B}$ un omomorfismo di anelli. Si dice che $f$ soddisfa il \emph{going-up} se per ogni $\pp_1\comma\pp_2\in\Spec A\comma\qq_1\in\Spec B$ tali che $\pp_1\subs\pp_2$ e $\pp_1=f^*(\qq_1)$ esiste $\qq_2\in\Spec B$ tale che $\qq_1\subs\qq_2$ e $\pp_2=f^*(\qq_2)$.
$$
\begin{diagram}
\pp_1\arrow[r,symbol=\subs]&\pp_2\arrow[r,symbol=\subs]&A\arrow[d,"f"]\\
\qq_1\arrow[r,symbol=\subs]\arrow[u,mapsto,"f^*"]&\exists\qq_2\arrow[r,symbol=\subs]\arrow[u,mapsto,"f^*"]&B
\end{diagram}
$$
\end{definition}
\begin{definition}
Sia $\map{f}{A}{B}$ un omomorfismo di anelli. Si dice che $f$ soddisfa il \emph{going-down} se per ogni $\pp_1\comma\pp_2\in\Spec A\comma\qq_2\in\Spec B$ tali che $\pp_1\subs\pp_2$ e $\pp_2=f^*(\qq_2)$ esiste $\qq_1\in\Spec B$ tale che $\qq_1\subs\qq_2$ e $\pp_1=f^*(\qq_1)$.
$$
\begin{diagram}
\pp_1\arrow[r,symbol=\subs]&\pp_2\arrow[r,symbol=\subs]&A\arrow[d,"f"]\\
\exists\qq_1\arrow[r,symbol=\subs]\arrow[u,mapsto,"f^*"]&\qq_2\arrow[r,symbol=\subs]\arrow[u,mapsto,"f^*"]&B
\end{diagram}
$$
\end{definition}
\begin{proposition}
Sia $\map{f}{A}{B}$ un omomorfismo di anelli. Se $\map{f^*}{\Spec A}{\Spec B}$ è chiusa, allora $f$ soddisfa il \emph{going-up}.
\end{proposition}
\begin{proposition}
Sia $\map{f}{A}{B}$ un omomorfismo di anelli, $Y=\Spec B\comma\qq\in Y$.
\begin{itemize}
\item $Y_\qq=\bigcap_{a\not\in\qq}Y_a$
\item $f^*(Y_\qq)=\bigcap_{a\not\in\qq}f^*(Y_a)$
\end{itemize}
\end{proposition}
\begin{proposition}
Sia $\map{f}{A}{B}$ un omomorfismo di anelli. Se $\map{f^*}{\Spec A}{\Spec B}$ è aperta, allora $f$ soddisfa il \emph{going-down}.
\end{proposition}
\begin{proposition}
Sia $\map{f}{A}{B}$ piatta. Allora $\map{f^*}{\Spec B}{\Spec A}$ è aperta, e $f$ soddisfa il \emph{going-down}.
\end{proposition}






