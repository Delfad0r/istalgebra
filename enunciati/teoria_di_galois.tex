\section{Teoria di Galois}
\begin{setting}
Sia $A$ un dominio normale, $K$ il suo campo dei quozienti, $L\sups K$ un'estensione di Galois finita, $B=\clos{A}^L\comma X=\Spec A\comma Y=\Spec B\comma\map{\varphi}{Y}{X}$ la mappa indotta dall'inclusione $A\subs B$, $\qq\in Y\comma\pp=\varphi(\qq)$.
$$
\begin{diagram}
\pp\arrow[r,symbol=\subs]&A\arrow[r,symbol=\subs]\arrow[d,symbol=\subs]&K\arrow[d,symbol=\subs]\\
\qq\arrow[r,symbol=\subs]\arrow[u,mapsto,"\varphi"]&B\arrow[r,symbol=\subs]&L
\end{diagram}
$$
Sia $G=\Gal(L/K)\comma Y_\pp=\varphi^{-1}(\pp)$. Siano $\kappa(\pp)\comma \kappa(\qq)$ i campi dei quozienti, rispettivamente, di $A/\pp\comma B/\qq$.
\end{setting}
\begin{lemma}
\leavevmode
\begin{itemize}
\item Se $\sigma\in G$, allora $\sigma B=B$.
\item Se $\sigma\in G$, allora $\sigma\qq\in Y_\pp$.
\item L'azione di $G$ su $Y_\pp$ è transitiva.
\end{itemize}
\end{lemma}
\begin{definition}
Si dice gruppo di decomposizione di $\qq$ il gruppo
$$
G_\qq=\{\sigma\in G:\sigma\qq=\qq\}.
$$
\end{definition}
\begin{lemma}
Sia $L^\qq=L^{G_\qq}\comma B^\qq=B\cap L^\qq\comma\qq'=\qq\cap B^\qq\comma \kappa(\qq')$ il campo del quozienti di $B^{\qq}/\qq'$.
\begin{itemize}
\item $L^\qq$ è il minimo campo $E$ tale che $K\subs E\subs L$ e $\qq$ è l'unico primo di $B$ sopra $\qq\cap E\in\Spec(B\cap E)$.
\item L'inclusione naturale $\kappa(\pp)\hookrightarrow\kappa(\qq')$ è un isomorfismo.
\end{itemize}
\end{lemma}
\begin{proposition}
L'estensione $\kappa(\pp)\hookrightarrow \kappa(\qq)$ è normale e i polinomi minimi degli elementi di $k(\qq)$ hanno grado uniformemente limitato.
\end{proposition}
\begin{proposition}
Sia $\map{\pi}{G_\qq}{\Gal(\kappa(\qq)/\kappa(\pp))}$ la mappa derivante dall'azione di $G_\qq$ su $B/\qq$ (e dunque su $\kappa(\qq)$). Allora $\pi$ è suriettiva.
\end{proposition}
\begin{lemma}
Sia $G$ un gruppo, $L$ un campo,
$$
\map{\chi_1,\ldots\chi_n}{G}{L^*}
$$
omomorfismi di gruppi a due a due distinti. Allora $\chi_1\comma\ldots\comma\chi_n$ sono $L$-linearmente indipendenti.
\end{lemma}
\begin{proposition}
Se $A$ è noetheriano, allora l'estensione $A\subs B$ è finita.
\end{proposition}
\begin{setting}
Supponiamo d'ora in poi che $A$ sia noetheriano con $\dim A=1$.
\end{setting}
\begin{definition}
Si chiama indice di inerzia l'intero
$$
f_\pp=[\kappa(\qq):\kappa(\pp)]
$$
(non dipende da $\qq$).
\end{definition}
\begin{definition}
$A_\pp$ e $B_\qq$ sono domini normali locali noetheriani di dimensione 1, dunque sono domini di valutazione discreta, diciamo $v_A$ e $v_B$. Siano $x\in A_\pp\comma y\in B_\qq$ generatori rispettivamente degli ideali massimali $\pp A_\pp\comma\qq B_\qq$. Supponiamo senza perdita di generalità che $v_A(x)=1$ e $v_B|_A=v_A$. Si chiama ramificazione di $\pp$ l'intero $e_\pp=(v_B(y))^{-1}$ (non dipende da $\qq$).
\end{definition}

\begin{proposition}
$$
[L:K]=\frac{\card{G}}{\card{G_\qq}}f_\pp e_\pp
$$
\end{proposition}


