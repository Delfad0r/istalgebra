\section{Dimensione}
\begin{definition}
Sia $A$ un anello. Si chiama dimensione di $A$ l'intero (eventualmente $+\infty$)
$$
\dim A=\sup\{n\in\NN:\text{esistono $\pp_0\,\ldots\pp_n\in\Spec A$ con $\pp_0\subsn\ldots\subsn\pp_n$}\}.
$$
\end{definition}
\begin{proposition}
Sia $A\subs B$ un'estensione intera. Allora $\dim A=\dim B$.
\end{proposition}
\begin{proposition}
Sia $A=\KK[y_1,\ldots,y_n]$ una $\KK$-algebra finitamente generata. Allora esistono $x_1,\ldots,x_m\in A$ algebricamente indipendenti con $m\le n$ tali che l'estensione $\KK[x_1,\ldots,x_m]\subs A$ è finita. Inoltre se $y_1,\ldots,y_n$ sono algebricamente dipendenti si può scegliere $m<n$.
\end{proposition}
\begin{proposition}
Sia $A=\KK[x_1,\ldots,x_n]$ anello di polinomi, $f\in A\setminus\{0\}$. Allora
$$
\dim A=\dim A_f=n.
$$
\end{proposition}
\begin{corollary}
Ogni $\KK$-algebra finitamente generata ha dimensione finita.
\end{corollary}
\begin{definition}
Sia $A$ un dominio e una $\KK$-algebra. Si dice base di trascendenza di $A$ su $\KK$ un insieme massimale di elementi di $A$ algebricamente indipendenti su $\KK$.
\end{definition}
\begin{definition}
Sia $A$ un dominio e una $\KK$-algebra. Si chiama grado di trascendenza di $A$ il cardinale
$$
\trdeg A=\min\{\#\mathcal{B}:\text{$\mathcal{B}$ è una base di trascendenza di $A$ su $\KK$}\}.
$$
\end{definition}
\begin{proposition}
Sia $\KK\subs K$ un'estensione di campi, con $\trdeg K$ finito. Allora ogni base di trascendenza di $K$ su $\KK$ ha cardinalità $\trdeg K$.
\end{proposition}
\begin{corollary}
Sia $A$ una $\KK$-algebra finitamente generata e un dominio, e sia $K$ il suo campo dei quozienti. Allora $\dim A=\trdeg K$.
\end{corollary}



