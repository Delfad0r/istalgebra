\section{Anelli Noetheriani e Artiniani}

\begin{definition}
Sia $M$ un $A$-modulo. Si chiama lunghezza di $M$ l'intero (eventualmente $+\infty$)
$$
\ell(M)=\sup\{n\in\NN:\text{esistono $M_0\subsn\ldots\subsn M_n$ sottomoduli di $M$}\}
$$
\end{definition}
\begin{lemma}
Sia $M$ un $A$-modulo.
\begin{itemize}
\item Se esiste una catena massimale di sottomoduli di $M$ di lunghezza $n$, allora tutte le catene massimali di sottomoduli hanno lunghezza $n$.
\item $\ell(M)<+\infty$ se e solo se $M$ è artiniano e noetheriano.
\end{itemize}
\end{lemma}
\begin{proposition}
Sia $A$ un anello. Allora $A$ è artiniano se e solo se è noetheriano e ha dimensione 0.
\end{proposition}
\begin{definition}
Sia $\kk$ un campo. Si dice valutazione una funzione $\map{v}{\kk^*}{\QQ}$ che soddisfa le seguenti proprietà per ogni $x,y\in\kk$:
\begin{enumerate}
\item $v(xy)=v(x)+v(y)$;
\item $v(x-y)\ge\min\{v(x),v(y)\}$.
\end{enumerate}
$v$ si dice discreta se $\im v\subs\ZZ q$ per un qualche $q\in\QQ$. Si pone convenzionalmente $v(0)=+\infty$.
\end{definition}
\begin{definition}
Un dominio $A$ si dice di valutazione se esiste una valutazione $\map{v}{K}{\QQ}$ (dove $K$ è il campo dei quozienti di $A$) tale che 
$$
A=\{x\in K:v(x)\ge0\}.
$$
A si dice di valutazione discreta se $v$ è discreta.
\end{definition}
\begin{proposition}
Sia $A$ un dominio di valutazione, e sia
$$
\mm=\{x\in A:v(x)>0\}.
$$
\begin{itemize}
\item $(A,\mm)$ è locale.
\item Se $v$ è discreta, sia $x\in A$ tale che $v(x)>0$ è minimo possibile. Allora tutti gli ideali non nulli di $A$ sono della forma $(x^n)$ per un qualche $n\in\NN$.
\end{itemize}
\end{proposition}
\begin{proposition}
Sia $(A,\mm)$ un anello locale noetheriano con $\dim A=1$. Allora sono equivalenti:
\begin{enumerate}
\item $A$ è un dominio di valutazione discreta;
\item $A$ è un dominio normale;
\item $\mm$ è principale.
\end{enumerate}
\end{proposition}