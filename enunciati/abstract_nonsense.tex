\section{Abstract Nonsense}
\begin{definition}
Una categoria $\mathscr{C}$ consta di:
\begin{itemize}
\item una classe $\Ob(\mathscr{C})$ di oggetti (spesso abbreviata $\mathscr{C}$);
\item per ogni $X,Y\in\Ob(\mathscr{C})$ un insieme $\Hom_{\mathscr{C}}(X,Y)$ di morfismi (spesso abbreviato $\Hom(X,Y)$);
\item per ogni $X,Y,Z\in\Ob(\mathscr{C})$ una legge di composizione
$$
\circ:\Hom_{\mathscr{C}}(Y,Z)\times\Hom_{\mathscr{C}}(X,Y)\longrightarrow\Hom_{\mathscr{C}}(X,Z).
$$
\end{itemize}
La legge di composizione soddisfa le seguenti proprietà:
\begin{itemize}
\item è associativa, ovvero per ogni $f\in\Hom_\mathscr{C}(X,Y)\comma g\in\Hom_\mathscr{C}(Y,Z)\comma h\in\Hom_\mathscr{C}(Z,W)$ vale $h\circ(g\circ f)=(h\circ g)\circ f$;
\item per ogni $X\in\Ob(\mathscr{C})$ esiste un morfismo $\1_X\in\Hom_{\mathscr{C}}$ tale che per ogni $f\in\Hom_\mathscr{C}(X,Y)\comma g\in\Hom_\mathscr{C}(Y,X)$ vale $f\circ\1_X=f\comma\1_X\circ g=g$.
\end{itemize}
\end{definition}
\begin{definition}
Sia $\mathscr{C}$ una categoria. Si chiama categoria opposta di $\mathscr{C}$ la categoria $\mathscr{C}^{\op}$ così definita:
\begin{itemize}
\item $\Ob(\mathscr{C}^{\op})=\Ob(\mathscr{C})$;
\item per ogni $X,Y\in\Ob(\mathscr{C}^{\op})$, $\Hom_{\mathscr{C}^{\op}}(X,Y)=\Hom_{\mathscr{C}}(Y,X)$;
\item per ogni $f\in\Hom_\mathscr{C^{\op}}(X,Y)\comma g\in\Hom_\mathscr{C^{\op}}(Y,Z)$, $g\circ^{\op}f=f\circ g$.
\end{itemize}
\end{definition}
\begin{definition}
Siano $\mathscr{C},\mathscr{D}$ due categorie. Un funtore (covariante) $\map{F}{\mathscr{C}}{\mathscr{D}}$ è una mappa che associa:
\begin{itemize}
\item a ogni oggetto $X\in\mathscr{C}$ un oggetto $F(X)\in\mathscr{D}$;
\item a ogni morfismo $f\in\Hom_\mathscr{C}(X,Y)$ un morfismo $F(f)\in\Hom_\mathscr{D}(F(X),F(Y))$.
\end{itemize}
Un funtore deve soddisfare le seguenti proprietà:
\begin{itemize}
\item per ogni $X\in\mathscr{C}$ vale $F(\1_X)=\1_{F(X)}$;
\item per ogni $f\in\Hom_\mathscr{C}(X,Y)\comma g\in\Hom_\mathscr{C}(Y,Z)$ vale $F(g\circ f)=F(g)\circ F(f)$.
\end{itemize}
\end{definition}
\begin{definition}
Siano $\mathscr{C},\mathscr{D}$ due categorie. Un funtore controvariante $\map{F}{\mathscr{C}}{\mathscr{D}}$ è un funtore covariante da $\mathscr{C}$ in $\mathscr{D}^{\op}$.
\end{definition}
\begin{definition}
Siano $\map{F,G}{\mathscr{C}}{\mathscr{D}}$ due funtori. Si chiama trasformazione naturale da $F$ in $G$ una collezione di morfismi $\eta=\{\eta_X\}_{X\in\mathscr{C}}$ con $\eta_X\in\Hom_\mathscr{D}(F(X),G(X))$ tale che per ogni $f\in\Hom_\mathscr{C}(X,Y)$ il seguente diagramma commuta
$$
\begin{diagram}
F(X)\ar[r,"\eta_X"]\ar[d,"F(f)"]&G(X)\ar[d,"G(f)"]\\
F(Y)\ar[r,"\eta_Y"]&G(Y)
\end{diagram}
$$
\end{definition}
\begin{definition}
Siano $\mathscr{C},\mathscr{D}$ due categorie. La categoria $\cat{Fun}(\mathscr{C},\mathscr{D})$ è così definita:
\begin{itemize}
\item $\Ob(\cat{Fun}(\mathscr{C},\mathscr{D}))$ è la classe dei funtori da $\mathscr{C}$ in $\mathscr{D}$;
\item $\Hom(F,G)$ è l'insieme delle trasformazioni naturali da $F$ in $G$;
\item per ogni $\eta=\{\eta_X\}_{X\in\mathscr{C}}\in\Hom(F,G)\comma\tau=\{\tau_X\}_{X\mathscr{C}}\in\Hom(G,H)$,
$$
\tau\circ\eta=\{\tau_X\circ\eta_X\}_{X\in\mathscr{C}}.
$$
\end{itemize}
Si definisce inoltre $\cat{Fun}^*(\mathscr{C},\mathscr{D})=\cat{Fun}(\mathscr{C},\mathscr{D}^{\op})$.
\end{definition}
\begin{proposition}
Sia $\mathscr{C}$ una categoria, $X\in\mathscr{C}$ un oggetto, $F\in\cat{Fun}^*(\mathscr{C},\cat{Set})$ un funtore controvariante. Poniamo $h_X=\Hom(\--,X)$. Allora esiste un isomorfismo $\Hom(h_X,F)\iso F(X)$ naturale in $F$ e in $X$.
\end{proposition}
\begin{corollary}
Siano $X,Y\in\mathscr{C}$ oggetti. Allora esiste un isomorfismo $\Hom(h_X,h_Y)\iso\Hom(X,Y)$ naturale in $X$ e in $Y$.
\end{corollary}
\begin{definition}
Siano $I,\mathscr{C}$ categorie, $\map{F}{I}{\mathscr{D}}$ un funtore. Si chiama limite inverso o proiettivo (e si indica con $\plim F(i)$) un oggetto $L\in\mathscr{C}$  dotato di morfismi $\varphi_i\in\Hom(L,F(i))$ per $i\in I$ con le seguenti proprietà:
\begin{enumerate}
\item per ogni morfismo $f\in\Hom_I(i,j)$ il seguente diagramma commuta
$$
\begin{diagram}
\phantom{}&L\ar[dl,"\varphi_i"]\ar[dr,"\varphi_j"]\\
F(i)\ar[rr,"F(f)"]&&F(j)
\end{diagram}
$$
\item per ogni oggetto $L'\in\mathscr{C}$ (dotato di morfismi $\varphi_i'\in\Hom(L',F(i))$ per $i\in I$) che soddisfa la (i) esiste un unico morfismo $\varphi\in\Hom(L',L)$ che fa commutare il seguente diagramma per ogni $f\in\Hom_I(i,j)$
$$
\begin{diagram}
\phantom{}&L'\ar[ddl,bend right=30,"\varphi_i'"]\ar[ddr,bend left=30,"\varphi_j'"]\ar[d,"\exists!\varphi"]\\
\phantom{}&L\ar[dl,"\varphi_i"]\ar[dr,"\varphi_j"]\\
F(i)\ar[rr,"F(f)"]&&F(j)
\end{diagram}
$$
\end{enumerate}
\end{definition}
\begin{definition}
Siano $I,\mathscr{C}$ categorie, $\map{F}{I}{\mathscr{D}}$ un funtore. Si chiama limite diretto o colimite (e si indica con $\dlim F(i)$ un oggetto $L\in\mathscr{C}$  dotato di morfismi $\psi_i\in\Hom(F(i),L)$ per $i\in I$ con le seguenti proprietà:
\begin{enumerate}
\item per ogni morfismo $f\in\Hom_I(i,j)$ il seguente diagramma commuta
$$
\begin{diagram}
\phantom{}&L\\
F(i)\ar[rr,"F(f)"]\ar[ur,"\psi_i"]&&F(j)\ar[ul,"\psi_j"]
\end{diagram}
$$
\item per ogni oggetto $L'\in\mathscr{C}$ (dotato di morfismi $\psi_i'\in\Hom(F(i),L')$ per $i\in I$) che soddisfa la (i) esiste un unico morfismo $\psi\in\Hom(L,L')$ che fa commutare il seguente diagramma per ogni $f\in\Hom_I(i,j)$
$$
\begin{diagram}
\phantom{}&L'\\
\phantom{}&L\ar[u,"\psi"]\\
F(i)\ar[rr,"F(f)"]\ar[ur,"\psi_i"]\ar[uur,bend left=30,"\psi_i'"]&&F(j)\ar[ul,"\psi_j"]\ar[uul,bend right=30,"\psi_j'"]
\end{diagram}
$$
\end{enumerate}
\end{definition}
\begin{definition}
Sia $\mathscr{C}$ una categoria, $I$ una categoria in cui gli unici morfismi sono le identità. Sia $\map{F}{I}{\mathscr{C}}$ un funtore.
Si chiama prodotto degli $F(i)$ (e si indica con $\prod_{i\in I}F(i)$) l'oggetto di $\mathscr{C}$ (se esiste) $\plim F(i)$.
\end{definition}
\begin{definition}
Sia $\mathscr{C}$ una categoria, $I$ una categoria con due oggetti $*_1$ e $*_2$ in cui gli unici morfismi oltre alle identità $\1_{*_1}\comma\1_{*_2}$ vanno da $*_1$ in $*_2$. Sia $\map{F}{I}{\mathscr{C}}$ un funtore, $X=F(*_1)\comma Y=F(*_2)$,
$$
\Phi=F(\Hom_I(*_1,*_2))\subs\Hom_{\mathscr{C}}(X,Y).
$$
Si chiama equalizzatore di $\Phi$ (e si indica con $\Eq\Phi$) l'oggetto di $\mathscr{C}$ (se esiste) $\plim F(i).$
\end{definition}
\begin{proposition}
Siano $I,\mathscr{C}$ categorie, $\map{F}{I}{\mathscr{C}}$ un funtore. Supponiamo esista il prodotto $X=\prod_{i\in I}F(i)$, e siano $\map{\pi_i}{X}{F(i)}$ le proiezioni. Per ogni $\varphi\in\Hom_I(i,j)$ sia $T_\varphi\in\Hom_\mathscr{C}(X,X)$ tale che
$$
\pi_k\circ T_\varphi=
\begin{cases}
\pi_k&\text{se $k\neq j$}\\
F(\varphi)\circ\pi_i&\text{se $k=j$}
\end{cases}.
$$
Allora
$$
\plim F(i)=\Eq\{T_\varphi\}.
$$
\end{proposition}
\begin{proposition}
Consideriamo il diagramma
$$
\begin{diagram}
\phantom{}&0\dar&0\dar&0\dar\\
\ldots\rar&A_{n+1}\rar\dar&A_n\rar\dar&A_{n-1}\rar\dar&\ldots\\
\ldots\rar&B_{n+1}\rar\dar&B_n\rar\dar&B_{n-1}\rar\dar&\ldots\\
\ldots\rar&C_{n+1}\rar\dar&C_n\rar\dar&C_{n-1}\rar\dar&\ldots\\
\phantom{}&0&0&0
\end{diagram}
$$
in cui le colonne sono esatte. Siano $\hat{A}=\plim A_n\comma\hat{B}=\plim B_n\comma\hat{C}=\plim C_n$. Allora la successione
$$
\begin{diagram}
0\rar&\hat{A}\rar&\hat{B}\rar&\hat{C}
\end{diagram}
$$
è esatta. Inoltre se le mappe $A_{n+1}\rightarrow A_n$ sono tutte suriettive allora la successione
$$
\begin{diagram}
\hat{B}\rar&\hat{C}\rar&0
\end{diagram}
$$
è esatta.
\end{proposition}


