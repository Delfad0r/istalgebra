\section{Abstract Nonsense II}
\begin{definition}
Sia $\mathscr{C}$ una categoria, $X,Y\in\mathscr{C}\comma\varphi\in\Hom(X,Y)$.
\begin{itemize}
\item $\varphi$ si dice monomorfismo se per ogni $Z\in\mathscr{C}\comma\alpha,\beta\in\Hom(Z,X)$ tali che $\varphi\circ\alpha=\varphi\circ\beta$ vale $\alpha=\beta$.
\item $\varphi$ si dice epimorfismo se per ogni $Z\in\mathscr{C}\comma\alpha,\beta\in\Hom(Y,Z)$ tali che $\alpha\circ\varphi=\beta\circ\varphi$ vale $\alpha=\beta$.
\end{itemize}
\end{definition}
\begin{definition}
Sia $\mathscr{C}$ una categoria, $A$ un anello. $\mathscr{C}$ si dice $A$-lineare se:
\begin{itemize}
\item per ogni $X,Y\in\mathscr{C}$, $\Hom(X,Y)$ è un $A$-modulo;
\item $\map{\circ}{\Hom(Y,Z)\times\Hom(X,Y)}{\Hom(X,Z)}$ è $A$-bilineare;
\item esiste $0\in\mathscr{C}$ tale che per ogni $X\in\mathscr{C}$ vale $\Hom(X,0)=\Hom(0,X)=0$;
\end{itemize}
$\mathscr{C}$ si dice $A$-additiva se per ogni $X,Y\in\mathscr{C}$ esiste il coprodotto $X\copr Y$.
\end{definition}
\begin{proposition}
Sia $\mathscr{C}$ una categoria $A$-lineare, $X,Y\in\mathscr{C}$. Allora $X\copr Y=X\pr Y$
\end{proposition}
\begin{definition}
Sia $\mathscr{C}$ una categoria $A$-lineare, $X,Y\in\mathscr{C}\comma\varphi\in\Hom(X,Y)$. Si chiama nucleo di $\varphi$ (e si indica con $\ker\varphi$) un oggetto $Z\in\mathscr{C}$ dotato di un morfismo $i\in\Hom(Z,X)$ con le seguenti proprietà:
\begin{enumerate}
\item $\varphi\circ i=0$;
\item per ogni oggetto $Z'\in\mathscr{C}$ dotato di un morfismo $i'\in\Hom(Z',X)$ tale che $\varphi\circ i'=0$ esiste un unico morfismo $j\in\Hom(Z',Z)$ che fa commutare il diagramma
$$
\begin{diagram}
Z\rar["i"]&X\rar["\varphi"]&Y\\
&Z'\uar["i'"]\arrow[lu,"\exists!j"]
\end{diagram}
$$
\end{enumerate}
\end{definition}
\begin{definition}
Sia $\mathscr{C}$ una categoria $A$-lineare, $X,Y\in\mathscr{C}\comma\varphi\in\Hom(X,Y)$. Si chiama conucleo di $\varphi$ (e si indica con $\coker\varphi$) un oggetto $W\in\mathscr{C}$ dotato di un morfismo $p\in\Hom(Y,W)$ con le seguenti proprietà:
\begin{enumerate}
\item $p\circ\varphi=0$;
\item per ogni oggetto $W'\in\mathscr{C}$ dotato di un morfismo $p'\in\Hom(Y,W')$ tale che $p'\circ\varphi=0$ esiste un unico morfismo $q\in\Hom(W,W')$ che fa commutare il diagramma
$$
\begin{diagram}
X\rar["\varphi"]&Y\rar["p"]\dar["p'"]&W\\
&W'\arrow[ru,"\exists!q"]
\end{diagram}
$$
\end{enumerate}
\end{definition}
\begin{definition}
Una categoria $\ZZ$-additiva $\mathscr{C}$ si dice preabeliana se ogni morfismo ammette nucleo e conucleo.
\end{definition}
\begin{proposition}
Sia $\mathscr{C}$ una categoria preabeliana, $X,Y\in\mathscr{C}\comma\varphi\in\Hom(X,Y)$. Allora esiste un unico morfismo $\bar\varphi$ che fa commutare il diagramma
$$
\begin{diagram}
\ker\varphi\rar["i"]&X\rar["\varphi"]\dar&Y\rar["p"]&\coker\varphi\\
&\coker i\rar["\exists!\bar\varphi"]&\ker p\uar
\end{diagram}
$$
\end{proposition}
\begin{definition}
Una categoria preabeliana $\mathscr{C}$ si dice abeliana se per ogni morfismo $\varphi$ il morfismo canonico $\bar\varphi$ è un isomorfismo. In tal caso si chiama immagine di $\varphi$ (e si indica con $\im\varphi$) l'oggetto $\coker i$ (o equivalentemente $\ker p$).
\end{definition}
\begin{proposition}
Sia $\mathscr{C}$ una categoria abeliana, $X,Y\in\mathscr{C}\comma\varphi\in\Hom(X,Y)$. Allora $\varphi$ è un isomorfismo se e solo se $\ker\varphi=\coker\varphi=0$.
\end{proposition}


