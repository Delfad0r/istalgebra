\documentclass[a4paper]{article}

\usepackage{mystyle}
\usepackage{subcaption}
\usepackage{titlesec}

\newcommand{\subsectionbreak}{\clearpage}

\usepackage{fancyhdr}
\pagestyle{fancy}
\fancyhf{}
\lhead{\today}
\chead{\textbf{Esercizi IV}}
\rhead{Filippo Gianni Baroni}

\newcommand*{\mattwo}[4]{\begin{pmatrix}#1&#2\\#3&#4\end{pmatrix}}
\newcommand*{\del}{\partial}

\begin{document}
\subsection*{Esercizio 9.3}
In tutto lo svolgimento dell'esercizio ometteremo gli indici qualora non ci fossero rischi di ambiguità.

\begin{lemma*}
Sia 
$$
\begin{diagram}0\rar&X\rar{\varphi}&Y\rar{\psi}&Z\rar&0\end{diagram}
$$
una successione esatta di $A$-moduli, e siano $\map{\rho}{Y}{X}$, $\map{\sigma}{Z}{Y}$ omomorfismi tali che $\rho\sigma=0$, $\rho\varphi=\1$, $\psi\sigma=\1$. Allora $\varphi\rho+\sigma\psi=\1$.
\end{lemma*}
\begin{proof}
Mostriamo innanzitutto che $\ker\psi\cap\ker\rho=0$. Sia $y\in\ker\psi=\im\varphi$. Allora $y=\varphi x$ per un qualche $x\in X$. Se $y$ sta anche in $\ker\rho$, allora $0=\rho y=\rho\varphi x=x$, da cui $y=\varphi x=0$. Mostriamo ora che
$$
y-\varphi\rho y-\sigma\psi y\in\ker\psi\cap\ker\rho
$$
per ogni $y\in Y$. Da un lato,
$$
\psi(y-\varphi\rho y-\sigma\psi y)=\psi y-\psi\varphi\rho y-\psi\sigma\psi y=\psi y-\psi y=0.
$$
Dall'altro,
$$
\rho(y-\varphi\rho y-\sigma\psi y)=\rho y-\rho\varphi\rho y-\rho\sigma\psi y=\rho y-\rho y=0.
$$
Segue che $y-\varphi\rho y-\sigma\psi y=0$ per ogni $y\in Y$.
\end{proof}

Il triangolo 
$$
\begin{diagram} X^*\rar{\varphi}&Y^*\rar{\psi}&Z^*\rar{\chi}&X^*[1]\end{diagram}
$$
è distinto se e solo se il triangolo
$$
\begin{diagram} Z^*[-1]\rar{-\chi}&X^*\rar{\varphi}&Y^*\rar{\psi}&Z^*\end{diagram}
$$
è distinto.
Mostriamo dunque che esiste un isomorfismo di triangoli
$$
\begin{diagram}
Z^*[-1]\rar{-\chi}&X^*\rar{\varphi}&Y^*\rar{\psi}&Z^*\\
Z^*[-1]\rar{-\chi}\uar[equal]&X^*\rar{a}\uar[equal]&C(-\chi)\rar{b}\uar{f}&Z^*\uar[equal]
\end{diagram}
$$
dove, ricordiamo, $C(-\chi)^n=X^n\dirsum Z^n$, $ax=(x,0)$, $b(x,z)=z$. Indichiamo con $\del_X,\del_Y,\del_Z,\del_C$ i bordi di $X,Y,Z,C(-\chi)$ (ometteremo i pedici qualora non ci fossero rischi di ambiguità). Abbiamo
$$
\del_C=\mattwo{\del_X}{-\chi}{0}{\del_Z}.
$$
Definiamo
\begin{align*}
f^n:X^n\dirsum Z^n&\longrightarrow Y^n\\
(x,z)&\longmapsto\varphi x+\sigma z
\end{align*}
\begin{itemize}
\item \textbf{$f$ è un morfismo di complessi.} Siano $x\in X^n,z\in Z^n$. Allora
\begin{align*}
f\del(x,z)&=f(\del x-\chi z,\del z)\\
&=f(\del x+\rho\del\sigma z,\del z)\\
&=\varphi\del x+\varphi\rho\del\sigma z+\sigma\del z\\
&=\varphi\del x+(\1-\sigma\psi)\del\sigma z+\sigma\del z\\
&=\varphi\del x+\del\sigma z-\sigma\del\psi\sigma z+\sigma\del z\\
&=\varphi\del x+\del\sigma z-\sigma\del z+\sigma\del z\\
&=\del\varphi x+\del\sigma z\\
&=\del f(x,z).
\end{align*}
\item \textbf{$f$ fa commutare il diagramma.} La verifica è immediata.
\item \textbf{$f$ è un isomorfismo di complessi.} Definiamo
\begin{align*}
g^n:Y^n&\longrightarrow X^n\dirsum Z^n\\
y&\longmapsto(\rho y,\psi y)
\end{align*}
Abbiamo
$$
gf(x,z)=g(\varphi x+\sigma z)=(\rho\varphi x+\rho\sigma z,\psi\varphi x+\psi\sigma z)=(x,z)
$$
e
$$
fg(y)=f(\rho y,\psi y)=\varphi\rho y+\sigma\psi y=(\varphi\rho+\sigma\psi)y=y.
$$
Segue che $g$ è un morfismo di complessi ($g\del=g\del fg=gf\del g=\del g$) e pertanto è proprio il morfismo inverso di $f$.
\end{itemize}


\end{document}
