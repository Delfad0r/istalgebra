\documentclass[a4paper]{article}

\usepackage{mystyle}
\usepackage{subcaption}
\usepackage{titlesec}

\newcommand{\subsectionbreak}{\clearpage}

\usepackage{fancyhdr}
\pagestyle{fancy}
\fancyhf{}
\lhead{\today}
\chead{\textbf{Esercizi IV}}
\rhead{Filippo Gianni Baroni}

\newcommand*{\mattwo}[4]{\begin{pmatrix}#1&#2\\#3&#4\end{pmatrix}}
\newcommand*{\del}{\partial}
\DeclareMathOperator{\Kom}{Kom}

\begin{document}
\subsection{Esercizio 9.2}
Come visto in classe, la risoluzione iniettiva di un complesso in $\Kom^+$ è unica a meno di isomorfismo unico. Osserviamo che l'identità $\map{\1}{I^*}{I^*}$ e $\map{\varphi}{I^*}{J^*}$ sono risoluzioni iniettive di $I^*$. Dunque esiste un unico isomorfismo che fa commutare il diagramma
$$
\begin{diagram}
I^*\rar{\1}\arrow[dr,"\varphi"]&I^*\dar[dashed]\\
&J^*
\end{diagram}
$$
Ma l'unico morfismo che fa commutare il diagramma è $\varphi$, dunque $\varphi$ è un isomorfismo (nella categoria omotopica)

\subsection*{Esercizio 9.3}
In tutto lo svolgimento dell'esercizio ometteremo gli indici qualora non ci fossero rischi di ambiguità.

\begin{lemma*}
Sia 
$$
\begin{diagram}0\rar&X\rar{\varphi}&Y\rar{\psi}&Z\rar&0\end{diagram}
$$
una successione esatta di $A$-moduli, e siano $\map{\rho}{Y}{X}$, $\map{\sigma}{Z}{Y}$ omomorfismi tali che $\rho\sigma=0$, $\rho\varphi=\1$, $\psi\sigma=\1$. Allora $\varphi\rho+\sigma\psi=\1$.
\end{lemma*}
\begin{proof}
Mostriamo innanzitutto che $\ker\psi\cap\ker\rho=0$. Sia $y\in\ker\psi=\im\varphi$. Allora $y=\varphi x$ per un qualche $x\in X$. Se $y$ sta anche in $\ker\rho$, allora $0=\rho y=\rho\varphi x=x$, da cui $y=\varphi x=0$. Mostriamo ora che
$$
y-\varphi\rho y-\sigma\psi y\in\ker\psi\cap\ker\rho
$$
per ogni $y\in Y$. Da un lato,
$$
\psi(y-\varphi\rho y-\sigma\psi y)=\psi y-\psi\varphi\rho y-\psi\sigma\psi y=\psi y-\psi y=0.
$$
Dall'altro,
$$
\rho(y-\varphi\rho y-\sigma\psi y)=\rho y-\rho\varphi\rho y-\rho\sigma\psi y=\rho y-\rho y=0.
$$
Segue che $y-\varphi\rho y-\sigma\psi y=0$ per ogni $y\in Y$.
\end{proof}

Il triangolo 
$$
\begin{diagram} X^*\rar{\varphi}&Y^*\rar{\psi}&Z^*\rar{\chi}&X^*[1]\end{diagram}
$$
è distinto se e solo se il triangolo
$$
\begin{diagram} Z^*[-1]\rar{-\chi}&X^*\rar{\varphi}&Y^*\rar{\psi}&Z^*\end{diagram}
$$
è distinto.
Mostriamo dunque che esiste un isomorfismo di triangoli
$$
\begin{diagram}
Z^*[-1]\rar{-\chi}&X^*\rar{\varphi}&Y^*\rar{\psi}&Z^*\\
Z^*[-1]\rar{-\chi}\uar[equal]&X^*\rar{a}\uar[equal]&C(-\chi)\rar{b}\uar{f}&Z^*\uar[equal]
\end{diagram}
$$
dove, ricordiamo, $C(-\chi)^n=X^n\dirsum Z^n$, $ax=(x,0)$, $b(x,z)=z$. Indichiamo con $\del_X,\del_Y,\del_Z,\del_C$ i bordi di $X,Y,Z,C(-\chi)$ (ometteremo i pedici qualora non ci fossero rischi di ambiguità). Abbiamo
$$
\del_C=\mattwo{\del_X}{-\chi}{0}{\del_Z}.
$$
Definiamo
\begin{align*}
f^n:X^n\dirsum Z^n&\longrightarrow Y^n\\
(x,z)&\longmapsto\varphi x+\sigma z
\end{align*}
\begin{itemize}
\item \textbf{$f$ è un morfismo di complessi.} Siano $x\in X^n,z\in Z^n$. Allora
\begin{align*}
f\del(x,z)&=f(\del x-\chi z,\del z)\\
&=f(\del x+\rho\del\sigma z,\del z)\\
&=\varphi\del x+\varphi\rho\del\sigma z+\sigma\del z\\
&=\varphi\del x+(\1-\sigma\psi)\del\sigma z+\sigma\del z\\
&=\varphi\del x+\del\sigma z-\sigma\del\psi\sigma z+\sigma\del z\\
&=\varphi\del x+\del\sigma z-\sigma\del z+\sigma\del z\\
&=\del\varphi x+\del\sigma z\\
&=\del f(x,z).
\end{align*}
\item \textbf{$f$ fa commutare il diagramma.} La verifica è immediata.
\item \textbf{$f$ è un isomorfismo di complessi.} Definiamo
\begin{align*}
g^n:Y^n&\longrightarrow X^n\dirsum Z^n\\
y&\longmapsto(\rho y,\psi y)
\end{align*}
Abbiamo
$$
gf(x,z)=g(\varphi x+\sigma z)=(\rho\varphi x+\rho\sigma z,\psi\varphi x+\psi\sigma z)=(x,z)
$$
e
$$
fg(y)=f(\rho y,\psi y)=\varphi\rho y+\sigma\psi y=(\varphi\rho+\sigma\psi)y=y.
$$
Segue che $g$ è un morfismo di complessi ($g\del=g\del fg=gf\del g=\del g$) e pertanto è proprio il morfismo inverso di $f$.
\end{itemize}

\subsection*{Esercizio 9.5}
\begin{enumerate}[(1)]
\item Consideriamo la risoluzione proiettiva di $A/(f)$
$$
\begin{diagram}
\ldots\rar&0\rar&A\rar{\varphi}&A\rar&A/(f)\rar&0
\end{diagram}
$$
dove $\varphi$ è la moltiplicazione per $f$.

Per calcolare $\Ext^i(A/(f),M)$ applichiamo il funtore $\Hom(-,M)$, ottenendo il complesso
$$
\begin{diagram}
0\rar&\Hom(A,M)\rar{-\circ\varphi}&\Hom(A,M)\rar&0\rar&\ldots
\end{diagram}
$$
Sappiamo che $\Hom(A,M)$ è canonicamente isomorfo a $M$ mediante l'applicazione
\begin{align*}
\Hom(A,M)&\longrightarrow M\\
g&\longmapsto g(1)
\end{align*}
Il complesso diventa dunque
$$
\begin{diagram}
0\rar&M\rar{\tilde\varphi}&M\rar&0\rar&\ldots
\end{diagram}
$$
dove $\tilde\varphi$ è la moltiplicazione per $f$ (modulo $\mm$). Distinguiamo due casi.
\begin{itemize}
\item $f(0,0)=0$ (ovvero $f\in\mm$). In tal caso $\tilde\varphi$ è la mappa nulla, dunque calcolando la coomologia del complesso si ottiene
$$
\begin{cases}
\Ext^0(A/(f),M)=M\\
\Ext^1(A/(f),M)=M\\
\Ext^i(A/(f),M)=0&\text{per $i\ge 2$}.
\end{cases}
$$
\item $f(0,0)\neq0$ (ovvero $f\not\in\mm$). In tal caso $\tilde\varphi$ è un isomorfismo, dunque calcolando la coomologia del complesso si ottiene
$$
\Ext^i(A/(f),M)=0\qquad\text{per ogni $i\ge 0$}.
$$
\end{itemize}

Per calcolare $\Tor_i(A/(f),M)$ applichiamo il funtore $-\tensor_AM$ alla risoluzione proiettiva di $A/(f)$, ottenendo il complesso
$$
\begin{diagram}
\ldots\rar&0\rar&A\tensor_AM\rar{\varphi\tensor\1}&A\tensor_AM\rar&0
\end{diagram}
$$
Sappiamo che $A\tensor_AM$ è canonicamente isomorfo a $M$ mediante l'applicazione
\begin{align*}
A\tensor_AM&\longrightarrow M\\
a\tensor m&\longmapsto am
\end{align*}
Il complesso diventa dunque
$$
\begin{diagram}
\ldots\rar&0\rar&M\rar{\tilde\varphi}&M\rar&0
\end{diagram}
$$
Si presentano i medesimi due casi di prima.
\begin{itemize}
\item $f(0,0)=0$ (ovvero $f\in\mm$). In tal caso $\tilde\varphi$ è la mappa nulla, dunque calcolando l'omologia del complesso si ottiene
$$
\begin{cases}
\Tor_0(A/(f),M)=M\\
\Tor_1(A/(f),M)=M\\
\Tor_i(A/(f),M)=0&\text{per $i\ge 2$}.
\end{cases}
$$
\item $f(0,0)\neq0$ (ovvero $f\not\in\mm$). In tal caso $\tilde\varphi$ è un isomorfismo, dunque calcolando l'omologia del complesso si ottiene
$$
\Tor_i(A/(f),M)=0\qquad\text{per ogni $i\ge 0$}.
$$
\end{itemize}
\item Consideriamo la risoluzione proiettiva di $M$
$$
\begin{diagram}
\ldots\rar&0\rar&A\rar{\varphi}&A\dirsum A\rar{\psi}&A\rar&M\rar&0
\end{diagram}
$$
dove $\varphi(f)=(yf,xf)$ e $\psi(f,g)=xf-yg$ (si verifica facilmente che è una successione esatta).

Per calcolare $\Ext^i(M,A/(x))$ applichiamo il funtore $\Hom(-,A/(x))$, ottenendo il complesso
$$
\begin{diagram}
0\rar&\Hom(A,A/(x))\rar{-\circ\psi}&\Hom(A\dirsum A,A/(x))\arrow[dl,out=-5,in=175,"-\circ\varphi"]&\phantom{\ldots}\\
&\Hom(A,A/(x))\rar&0\rar&\ldots
\end{diagram}
$$
Sappiamo che $\Hom(A,A/(x))$ è canonicamente isomorfo a $A/(x)$ mediante l'applicazione
\begin{align*}
\Hom(A,A/(x))&\longrightarrow A/(x)\\
g&\longmapsto g(1)
\end{align*}
Il complesso diventa dunque
$$
\begin{diagram}
0\rar&A/(x)\rar{\tilde\psi}&A/(x)\dirsum A/(x)\rar{\tilde\varphi}&A/(x)\rar&0\rar&\ldots
\end{diagram}
$$
dove $\tilde\psi(f)=(xf,-yf)=(0,-yf)$ e $\tilde\varphi(f,g)=yf+xg=yf$. Possiamo ora calcolare la coomologia del complesso.
\begin{itemize}
\item $\tilde\psi$ è iniettiva, dunque $\Ext^0(M,A/(x))=0$.
\item L'immagine di $\tilde\psi$ è $0\dirsum(y)A/(x)$, mentre il nucleo di $\tilde\varphi$ è $0\dirsum A/(x)$, dunque $\Ext^1(M,A/(x))=A/(x,y)=M$.
\item L'immagine di $\tilde\varphi$ è $(y)A/(x)$, dunque $\Ext^2(M,A/(x))=A/(x,y)=M$.
\item Per $i\ge 3$ abbiamo $\Ext^i(M,A/(x))=0$.
\end{itemize}

\end{enumerate}



\end{document}
